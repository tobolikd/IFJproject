%Author: xtobol06

%class
\documentclass[a4paper, 12pt]{article}

%encoding
\usepackage[T1]{fontenc}           %font encoding
\usepackage[utf8]{inputenc}         %script encoding

\usepackage[bibencoding=UTF-8,style=iso-numeric]{biblatex}
\addbibresource{sources.bib}

%packages
\usepackage[czech]{babel}           %language
\usepackage[a4paper, text={17cm,24cm}, left=2cm, top=3 cm]{geometry}		%layout
\usepackage{times}					%font
\usepackage[ruled, czech, linesnumbered, longend, noline]{algorithm2e}		%algorithms
\usepackage[unicode,hidelinks]{hyperref}	%links
\usepackage{amsmath}
\usepackage{tabularx}
\usepackage{multicol}
\usepackage{multirow}
\usepackage{graphicx}
\usepackage{float}
\usepackage{csquotes}

\begin{document}

    \begin{titlepage}
		\centering

        \includegraphics{src/fitlogo.pdf}

        \vspace{\stretch{0.382}}

        {\Huge Dokumentace k projektu z IFJ a IAL\\[0.4em]
            \LARGE Tým xbielg00, varianta TRP}

        \vspace{\stretch{0.618}}

        \begin{table}[H]
            \hfill
            \begin{tabularx}{0.5\textwidth}{Xr}
                \textbf{Gabriel Biel} & \textbf{xbielg00} \\
                Adam Gabrys & xgabry01 \\
                Jakub Mikyšek & xmikys03 \\
                David Tobolík & xtobol06 \\
            \end{tabularx}
        \end{table}
	\end{titlepage}

    \tableofcontents
    \newpage

    \section{Týmová spolupráce}
        \subsection{Komunikace v týmu}
            Vzájemná domluva a komunikace probíhala online i offline. A to prostřednictvím Discord serveru - pro převážně obecnější problematiky, pro ty složitější a již více konkrétní nám sloužila funkce GitHub - Issues, kde jsme mohli jednodušše diskutovat o již kontrétní části kódu. K tomu jsme se každý čtvrtek setkávali na schůzích uskutečněných v budově škole, kde jsme řešili ty nejzásadnější problémy a rozhodnutí z hlediska směřování.
        \subsection{Verzovací systém}
            Při výběru verzovacího systému pro nás bylo jasnou volbou zvolit populární a osvědčený Git s nahraváním na server GitHub, kde probíhala i již zmíněná vzájemná komunikace.
        \subsection{Testování}
            Pro ověření správnosti jsme používali framework GoogleTest v jazyce C++, který nám sloužil k unit testování.
        \subsection{Rozdělení práce}
            \begin{itemize}
                \item Gabriel Biel (xbielg00)
                    \begin{itemize}
                        \item architektura compileru, lexikální analýza, precedenční tabulka, precedenční analýza, testy pro precedenční analýzu, rozhraní pro tabulku symbolů, funkce v generátoru kódu
                    \end{itemize}
                \item Adam Gabrys (xgabry01)
                    \begin{itemize}
                        \item
                    \end{itemize}
                \item Jakub Mikyšek (xmikys03)
                    \begin{itemize}
                        \item gramatika, rekurzivní syntaktická analýza, sémantické kontroly pro proměnné, dokumentace
                    \end{itemize}
                \item David Tobolík (xtobol06)
                    \begin{itemize}
                        \item sémantické kontroly pro volání funkce, datové struktury pro ASS, zásobníky, struktura projektu, cmake a struktura testů, testy (lexikální analýza, AST, zásobníky), code review, makefile, dokumentace
                    \end{itemize}
            \end{itemize}

    \section{Popis fungování a struktura překladače}

    \section{Lexikální analýza}

    \section{Syntaktická analýza}
    Rekurzivní sestup se volá funkcí \texttt{synAnalyser} a je implementovaný v souborech \texttt{synAnalyzer.c} a \texttt{synAnalyzer.h}. Řídí se podle navrženené gramatiky (viz \ref{}), kde každé pravidlo představuje vlastní funkci. Funkce postupně prochází tokeny získané od lexikální analýzy, kontroluje jestli je vše gramaticky správně, naplňuje tabulku symbolů a tvoří abstraktní syntaktický strom (dále ASS). Při nalezení výrazu volá funkci \texttt{parseExpression}, která spouští precedenční analýzu. Při nalezení chyby se nastaví příslušný \texttt{errorCode} a chyba se v rekurzi propaguje dál, aby program ukončil kontrolu. Funkce vrací vytvořený ASS, naplněnou tabulku symbolů a informaci o tom, jestli rekurze proběhla syntaticky a sémanticky správně či nikoliv.

    \subsection{Rekurzivní sestup}

    \subsection{Precedenční analýza}

    \begin{table}[h]
    \centering
    \begin{tabular}[p]{| l | c | c | c | c | c | c | c | c | c | c | c | c | c | c | c | c |}
        \hline
        &   \textbf{$\times$} & \textbf{$/$} & \textbf{$+$} & \textbf{$-$} & \textbf{$\cdot$} & \textbf{$<$} & \textbf{$>$} & \textbf{{$\geq$}} & \textbf{$\leq$} & \textbf{$\neq$} & \textbf{$=$} & \textbf{$($} & \textbf{$)$} & \textbf{$i$} & \textbf{\$} \\
        \hline
        \textbf{$\times$} &
            $>$ & $>$ & $>$ & $>$ & $>$ & $>$ & $>$ & $>$ & $>$ & $>$ & $>$ & $<$ & $>$ & $<$ & $>$ \\
        \textbf{$/$} &
            $>$ & $>$ & $>$ & $>$ & $>$ & $>$ & $>$ & $>$ & $>$ & $>$ & $>$ & $<$ & $>$ & $<$ & $>$ \\
        \textbf{$+$} &
            $<$ & $<$ & $>$ & $>$ & $>$ & $>$ & $>$ & $>$ & $>$ & $>$ & $>$ & $<$ & $>$ & $<$ & $>$ \\
        \textbf{$-$} &
            $<$ & $<$ & $>$ & $>$ & $>$ & $>$ & $>$ & $>$ & $>$ & $>$ & $>$ & $<$ & $>$ & $<$ & $>$ \\
        \textbf{$\cdot$} &
            $<$ & $<$ & $>$ & $>$ & $>$ & $>$ & $>$ & $>$ & $>$ & $>$ & $>$ & $<$ & $>$ & $<$ & $>$ \\
        \textbf{$<$} &
            $<$ & $<$ & $<$ & $<$ & $<$ &     &     &     &     & $>$ & $>$ & $<$ & $>$ & $<$ & $>$ \\
        \textbf{$>$} &
            $<$ & $<$ & $<$ & $<$ & $<$ &     &     &     &     & $>$ & $>$ & $<$ & $>$ & $<$ & $>$ \\
        \textbf{$\geq$} &
            $<$ & $<$ & $<$ & $<$ & $<$ &     &     &     &     & $>$ & $>$ & $<$ & $>$ & $<$ & $>$ \\
        \textbf{$\leq$} &
            $<$ & $<$ & $<$ & $<$ & $<$ &     &     &     &     & $>$ & $>$ & $<$ & $>$ & $<$ & $>$ \\
        \textbf{$\neq$} &
            $<$ & $<$ & $<$ & $<$ & $<$ & $<$ & $<$ & $<$ & $<$ &     &     & $<$ & $>$ & $<$ & $>$ \\
        \textbf{$=$} &
            $<$ & $<$ & $<$ & $<$ & $<$ & $<$ & $<$ & $<$ & $<$ &     &     & $<$ & $>$ & $<$ & $>$ \\
        \textbf{$($} &
            $<$ & $<$ & $<$ & $<$ & $<$ & $<$ & $<$ & $<$ & $<$ & $<$ & $<$ & $<$ & $=$ & $<$ &     \\ 
        \textbf{$)$} &
            $>$ & $>$ & $>$ & $>$ & $>$ & $>$ & $>$ & $>$ & $>$ & $>$ & $>$ &     & $>$ &     & $>$ \\ 
        \textbf{$i$} &
            $>$ & $>$ & $>$ & $>$ & $>$ & $>$ & $>$ & $>$ & $>$ & $>$ & $>$ &     & $>$ &     & $>$ \\ 
        \textbf{$\$$} &
            $<$ & $<$ & $<$ & $<$ & $<$ & $<$ & $<$ & $<$ & $<$ & $<$ & $<$ & $<$ &     & $<$ &     \\ 
        \hline
    \end{tabular}
    \caption{Precedenční tabulka.}
    \label{tabulka precedence}
\end{table}

    \section{Tabulka symbolů}

    \section{Abstraktní syntaktický strom}
    Abstraktní syntaktický strom (dále ASS) je implementovaný jako zásobník (viz \ref{}), který obsahuje strukturu pro ASS. Ta je tvořena typem a volitelnými daty.
    \begin{itemize}
        \item Aritmetické, řetezcový a relační operátory -- každý má svůj typ, nemají data. Na zásobníku předcházejí 2 operandy, nad kterými se má operace provést. Operandem může být konstanta, proměnná nebo další operátor, v tom případě se operace provede nad výsledkem operace, který je dán tímto operátorem.
        \item Přiřazení -- nemá data, na zásobníku následuje proměnná, do které se přiřazuje a po ní výraz nebo volání funkce, které se má přiřadit.
        \item Proměnná -- v datech obsahuje ukazatel do tabulky symbolů.
        \item Konstanty -- každý typ konstanty má svůj typ v ASS (int, string, float, null) a obsahuje, kromě typu null, hodnotu této konstanty.
        \item Deklarace funkce -- v datech je uložený ukazatel do tabulky symbolů na tuto funkci.
        \item Volání funkce -- v datech obsahuje ukazatel na strukturu volání funkce, v ní je uložen ukazatel do tabulky symbilů a seznam parametrů. Parametry mají rovněž typ (proměnná nebo konstanta -- int, string, float, null) a svoje data (ukazatel do tabulky symbolů nebo hodnota konstanty).
        \item Volání return -- typ bez návratové hodnoty nebo s návratovou hodnotou, v tom případě na zásobníku následuje výraz, který se má vrátit.
        \item Podmínky -- if, else, while -- nemají data, na zásobníku následuje výraz, který je v podmínce.
        \item Konec výrazu -- značí konec výrazu nebo volání funkce, slouží k rozpoznání konce postfixového zápisu.
        \item Konec složeného výrazu -- značí konec složeného příkazu ve složených závorkách (if, else, deklarace funkce).
    \end{itemize}
    Strom se ukládá od začátku programu, takže při generování kódu se musí zásobník číst odzadu (viz \ref{zasobniky} a \ref{generovani}).
    \section{Zásobníky} \label{zasobniky}
    Zásobníky byly potřeba na více místech (ASS, precedenční analýza, generování kódu, \ldots). Implementace zásobníků je tedy obecná pomocí makra, které je možné použít pro libovolný ukazatel. V některých aplikacích je potřeba přístup k zásobníku z obou stran, pro operace push a pop existují tedy 2 varianty -- pro přístup od začátku a od konce. Pro operaci pop je potřeba v makru specifikovat destruktor, který správně uvolní prvek, což je vhodné pro komplexnější strkutury.

    \section{Sémantické kontroly}
    Všechny sémantické kontroly probíhají s pomocí tabulky symbolů. Před spuštěním syntaktické analýzy, proběhne tzv. "první průchod" za pomocí funkce \texttt{fncDeclarationTable}, která kontroluje jen definice funkcí a nahrává je společně s jejich parametry a návratovým typem do tabulky symbolů. V návaznosti na tento první průchod se v precedenční analýze při volání funkcí provádí kontrola typu a počtu parametrů nebo zda byla funkce definována. Sémantické kontroly pro proměnné a jejich (ne)definování se provádějí v syntaktické a precedenční analýze.
    
    \section{Generování kódu} \label{generovani}
    Generování kódu zajišťuje funkce \texttt{codeGenerator}, která slouží jako kontroler celého generování. Předává si řízení s funkcemi, které slouží pro generování konkrétních částí, stará se o správu kontextu programu, uvolnění dat apod.

    \subsection{Kontext programu}
    Funkce \texttt{codeGenerator} vytváří kontext programu a předává ho některým funkcím. Kontext obsahuje čítač podmínek if, while, ukazatel do tabulky symbolů na aktuální deklarovanou funkci, zásobník složených výrazů a proměnnou pro určení čísla návěští else.

    Zásobník složených výrazů obsahuje struktury s typem složeného výrazu (deklarace funkce, if, else, while) a pořadové číslo. Pořadové číslo je určeno čítačem v kontextu programu, u deklarace funkce není potřeba.

    \subsection{Statické funkce a definice proměnných}
    Před spuštěním veškerého generování kódu je spuštěna funkce pro vygenerování statické části programu a definice proměnných. Statická část obsahuje definici globálních pomocných proměnných, které slouží místo registrů a definici vestavěných funkcí. Tato část je přeskočena instrukcí \texttt{JUMP} až na začátek programu. Na začátku se vytvoří nový lokální rámec a nadefinují se všechny proměnné, které jsou použity v hlavním těle programu (i v podmínkách, takže všude kromě deklarace funkcí).

    \subsection{}

    \newpage
	%bibliography
	\renewcommand{\refname}{Zdroje}
	\printbibliography
\end{document}

